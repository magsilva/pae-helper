\section{Corre��o de trabalhos}

\subsection{Introdu��o}

A corre��o � sempre otimista, sendo somente assinalado os erros. Na pior das hip�teses, o aluno ir� tirar mais nota do que o normal (que � a regra normal). Neste sentido ainda, determina-se a porcentagem \textbf{errada} do crit�rio (e n�o o usual, que � a porcentagem correta).

\subsection{Formatos}

corrigido-v0.py
Este � o primeiro formato de notas, consistindo somente de um n�mero inteiro, de 0 a 100, representando a nota do aluno em quest�o.

corrigido-v1.py
O erro pode ser total ou parcial, ou seja, quest�o inteira ou metade errada. O formato deste m�todo � "#" "ERRO" [ "TOTAL" | "PARCIAL" ] exercicio"."criterio. Por exemplo:
\begin{verbatim}
	# ERRO TOTAL 1.1
	# ERRO PARCIAL 2.7
\end{verbatim}

corrigido-v2.py
Todo erro � total. O formato deste m�todo � "#" "ERRO" exercicio criterio. Por exemplo:
\begin{verbatim}
	# ERRO 1 1
	# ERRO 2 7
\end{verbatim}

corrigido-v3.py
O erro pode ser total ou parcial, ou seja, quest�o inteira ou metade errada. O formato deste m�todo � "#" "ERRO" [ "TOTAL" | "PARCIAL" ] exercicio criterio. Ele � praticamente id�ntico ao \ref{}, apenas n�o possui um ponto separando o exerc�cio e o e crit�rio em quest�o. Por exemplo:
\begin{verbatim}
	# ERRO TOTAL 1 1
	# ERRO PARCIAL 2 7
\end{verbatim}


corrigido-v4.py
O erro variar entre 0 e 100 porcento do crit�rio. O formato deste m�todo � "#" "ERRO" exercicio criterio porcentagem. Por exemplo:
\begin{verbatim}
	# ERRO 1 1 100
	# ERRO 2 7 50
\end{verbatim}
